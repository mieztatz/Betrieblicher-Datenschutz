\documentclass[10pt, a4paper]{scrreprt}
\usepackage[german,ngerman]{babel}
\usepackage{graphicx}
\usepackage{float}
%\usepackage{lmodern}
\usepackage[utf8]{inputenc}
\setcounter{section}{0}

\usepackage{geometry}
\geometry{verbose,a4paper,tmargin=10mm,bmargin=10mm,lmargin=30mm,rmargin=30mm}

%% alle serifenlosen Teile des Dokuments in Helvetica gesetzt
\usepackage{helvet}

%% Zeilenabstand
\linespread{1.00}



%% Schriftformatierung der einzelnen Elemente wie Überschrift usw.
\setkomafont{chapter}{\sffamily \Large}
\setkomafont{section}{\sffamily \textit}
\setkomafont{subsection}{\sffamily \textit}


\begin{document}
	
	% Festlegung Art der Zitierung - Havardmethode: Abkuerzung Autor + Jahr
	\bibliographystyle{alphadin}
	
	%% Deckplatt generieren
	\begin{titlepage}
		\centering
		{\sffamily\huge Fachbereich 07 Informatik/Mathematik \par}
		\includegraphics[width=0.5\textwidth]{Hochschule_Muenchen_Logo.png}\par
		\vspace{1cm}
		{\sffamily\LARGE Praktikum Datenschutz und Datensicherheit Sommersemester 2016\par}
		\vspace{1cm}
		{\sffamily\Large Prof. Dr. Rainer W. Gerling\par}
		{\sffamily\Large Heidi Schuster\par}
		\vspace{2cm}
		{\LARGE Man-in-the-Middle\par}
		\vspace{1.5cm}
		{\LARGE Fabian Uhlmann\par}
		{\LARGE IF6\par}
		\vspace{0.5cm}
		{\LARGE Diana Irmscher\par}
		{\LARGE IF7\par}
		\vspace{1cm}
		{\large \today\par}
		% Ende der Titelseite
		\vfill
	\end{titlepage}

	%% Inhaltsverzeichnis
	\tableofcontents
	
	%% Worum geht es, kurze Erklärung
	\begin{abstract}
		Im Rahmen des Studiums Bachelor Informatik absolvieren wir (Fabian Uhlmann und Diana Irmscher) die zusätzliche Ausbildung zum betrieblichen Datenschutz an der Hochschule München.
		
		Das Thema Datenschutz und IT-Sicherheit ist in den letzten Jahren immer mehr in den Vordergrund getreten. Meldungen über Angriff wie z.B. auf Bundestag im Mai 2015 und ganz aktuell auch der Krypto-Trojaner Locky sind fast täglich in den Nachrichten vertreten.
		
		Wir haben das Thema "'Man-in-the-Middle"' gewählt, weil dieses Thema sehr spannend ausgearbeitet werden kann.
		Dabei werden wir erst darauf eingehen, wie sich der Angriff zusammensetzt, wo genau die rechtlichen Verstöße liegen und wie man sich vor solche Angriffen schützen kann.
		
		Das Thema haben wir aufgeteilt in zwei Unterthemen.
		Herr Uhlmann wird darauf eingehen, wie man Sicherheit eines Systems mit einem MITM-Angriff sehr effizient aushebeln kann.
		Frau Irmscher beschäftigt sich mit dem gezielten Angriff in TLS/SSL und in gesnifften Netzwerken.
		\flushright{München, \today}
		 
	\end{abstract}
	
	%% Fabian
	\chapter{Man-in-the-Middle}
	Dell und Lenovo haben demonstriert, dass man mit Man-in-the-Middle
	Angriffen die Sicherheit eines Systems sehr effizient aushebeln kann.
	Wie funktioniert ein derartiger Angriff und was kann man tun, um sich
	zu schützen.
	\section{}
	\subsection{} 
	\subsection{} 
	
	%% Diana
	\chapter{Man-in-the-Middle-Angriffe im geswitchten Netz}
	Es gibt Man-in-the-Middle-Angriffe nicht nur gegen SSL/TLS
	Verbindungen sondern auch gegen "normale" Netzwerkverbindungen.
	Sie werden von Angreifern eingesetzt, um in einem geswitchten Netz
	zu sniffen.
	\section{Test}
	\subsection{test} 
	\subsection{test} 

	%% Was so alles in einen Anhang kommt
	\appendix
	
	\bibliography{literatur}
	
	
\end{document}
\subsection{Zwischenfazit}
Beide Systemhersteller nutzten ihre Position in der Marktwirtschaft und das Vertrauen, welches ihnen die Kunden entgegenbringen, schamlos aus. Sie ließen dem Käufer im Glauben, dass er ein solides Gerät mit sicherer Software gekauft hätte. Doch ohne selbständige Gegenmaßnahmen des Kunden war dieser, mittels der von Lenovo und DELL selbst signierten Zertifikate, potentiellen Angreifern hilflos ausgesetzt. Die Käufer müssen auf der einen Seite für die Zukunft hoffen, dass die Gerätehersteller aus ihren Fehlern gelernt haben und wieder sichere Systeme herstellen bzw. verkaufen. Auf der anderen Seite kann und sollte jeder selbständig Kontrollen und Überprüfungen durchführen. Außerdem ist es ratsam, sich ein wenig mit den Systemen zu befassen mit denen man täglichen Umgang hat und wichtige Tätigkeiten, wie z. B. Online Banking, durchführt. Darunter zählt auch das regelmäßige Abrufen von Informationen über Sicherheitsneuigkeiten in bekannten Sicherheitsforen, z. B. \url{https://www.heise.de/security}
\\
\\
Gemäß dem Sprichwort „Vertrauen ist gut, Kontrolle ist besser.“, sollte jeder Benutzer, mit den verfügbaren Möglichkeiten, die verwendeten Systeme und Dienste immer wieder überprüfen.
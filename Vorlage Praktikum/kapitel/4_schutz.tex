\section{Schutzmöglichkeiten}
Bedauerlicher weiße haben Betroffene nicht all zu viele Möglichkeiten sich gegen eben beschriebene Sicherheitslücken zu schützen. 
Eine Option wäre der Kauf von neuen Systemen ohne vorinstalliertes Betriebssystem. Wenn der Anwender selbst ein Betriebssystem nach dem Kauf installiert, kann er mit großer Sicherheit davon ausgehen, dass sich keine ungewollte, vorinstallierte Software und/oder kein selbst signiertes Zertifikat auf dem neu angeschafften Gerät befindet. Jedoch schützt diese Option nicht für solchen Zertifikaten, die sich erst nachträglich installieren. Ein Beispiel ist das DSDTestProvider Zertifikat, welches sich beim Besuch der Support-Webseite von DELL installiert. Vorstellbar sind auch Szenarien, wo sich Zertifikate bei der Installation von Software mit einschleusen.
Eine andere Variante des Schutzes ist die regelmäßige Kontrolle der auf dem System installierten Zertifikate. Durch diese Maßnahme können auch nachträglich installierte Zertifikate entdeckt werden. Zugegeben, bei der heutigen Masse an vorhanden, vertrauenswürdigen Zertifizierungsstellen und zugehörigen Zertifikaten, ist es nicht leicht den Überblick zu behalten. Erschwerend kommt hinzu, dass viele Programme zusätzlich einen eigenen Zertifikatsspeicher besitzen. Dabei kann nicht immer davon ausgegangen werden, dass diese Speicher sich auf den Zertifikatsspeicher des Betriebssystems beziehen bzw. die gleichen Zertifikate beinhalten.
Um sicherzustellen, dass mit dem richtigen Server kommuniziert wird, d. h. es besteht keine Verbindung zu einer gefälschten Webseite, sollte man das Zertifikat prüfen, welches für die aktuelle Kommunikation verwendet wird. Jeder Browser zeigt neben der HTTPS-Adresse ein Schlosssymbol. Über dieses Symbol werden die Eigenschaften, des aktuell für diese verschlüsselte Verbindung verwendete Zertifikat, erreicht. Verschiedene Webseiten, z. B. https://globalsign.ssllabs.com, bieten eine Überprüfung der gewünschten Webseite an. Als Ergebnis zeigen sie unter anderem an, welches originale Zertifikat diese Webseite wirklich verwendet. Mittels dieser Information und der Zertifikats-Eigenschaften durch den Browser, ist jeder in der Lage zu vergleichen, ob die verschlüsselte Verbindung das richtige Zertifikat nutzt. Handelt es sich nicht um das korrekte, originale Zertifikat und gab es keine Fehlermeldung, so muss davon ausgegangen werden, dass keine direkte Verbindung zu der gewünschten Webseite besteht. Dies bedeutet, die verschlüsselte Verbindung wurde durch eine MITM Attacke aufgebrochen und die Daten werden mitgelesen oder sogar manipuliert. Nach Aufdeckung einer solchen Attacke, muss das gefälschte Zertifikat im System gesucht und anschließend gelöscht werden. Dabei ist zu beachten, dass es nicht immer ausreicht nur das entdeckte Zertifikat selbst zu löschen. Da Zertifikate auf der Basis einer Public Key Infrastruktur erstellt sind, können noch weitere Zertifikate für solch eine Attacke verantwortlich sein. Es sollten neben dem aufgespürten, gefälschten Zertifikat zusätzlich alle anderen Zertifikate, die sich in der hierarchischen Kette befinden, überprüft und ggf. entfernt werden. Im Lenovo Beispiel konnten mittels des gehackten, selbst signierten Root-Zertifikats weitere Zertifikate erstellt werden. Mit jedem dieser Zertifiakte war anschließend eine MITM Attacke möglich. Das Sicherheitsproblem war mit der Eliminierung des für die MITM Attacke verwendeten Zertifikats nicht gelöst. Erst nach erfolgreichem Entfernen des selbst signierten Root-Zertifikats konnten keine weiteren gefälschten Zertifikate erzeugt werden.
Beim DELL Vorfall schrieb Liam Tung auf der ZDNET-Webseite: „[...] das einfache Entfernen des eDELLRoot-Zertifikats aus dem Administrator und persönlichen Zertifikatsspeicher ist nicht genug, um den Nutzer zu schützen. Einige Nutzer haben in der Tat berichtet, dass das Zertifikat nach einem Neustart wieder aufgetaucht ist.“[3]  Ursache für die Neuinstallation ist das DELL-Programm „Dell Foundation Services“, das dieses Zertifikat verwendet. Erst mit der Deinstallation des Programms bzw. eines Plugins des Programms und der manuellen Löschung des DELL Zertifikats, ist das selbst signierte Zertifikat dauerhaft vom System erfolgreich entfernt und die Sicherheitslücke geschlossen. Liam Tung schrieb zur korrekten Entfernung folgendes: „Um es dauerhaft zu entfernen und um zu verhindern, dass es sich erneut installiert, müssen Nutzer das eDELL Plugin entfernen.“ Für die detaillierte Information um welches Plugin es sich genau handelt, nutzt er die Ergebnisse von den Duo Security Forschern Darren Kemp, Michail Davidov und Kyle Lady. „‘Dies kann vollbracht werden mit Hilfe der Löschung des Dell.Foundation.Agent.Plugins.eDell.dll Moduls vom System. Geschieht dies nicht, so kann es weiterhin zur Aussetzung dieser Sicherheitslücke kommen.‘, sagte Duo Security.“ [3] Er fügte noch einen wichtigen Hinweis von Duo Security hinzu: „‘Beachte, immer wenn sie ein Werksreset auf ihrem DELL System durchführen, wird dieses Zertifikat und das eDell Plugin auf dem System wieder hergestellt und sie müssen es erneut manuell entfernen‘ [...] " [3]
Die beiden Beispiele zeigen, dass es nicht ausreicht nur die Zertifikatsspeicher auf ungewöhnliche Eintragungen zu durchsuchen, sondern ebenfalls die Programmliste des Systems auf unerwartete oder ggf. unnötig installierte Programme zu kontrollieren.
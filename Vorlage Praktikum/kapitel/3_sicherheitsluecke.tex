\section{Sicherheitslücken bei bekannten Computer Herstellern}
Man sollte davon ausgehen können, dass die großen PC-Hersteller selbst am besten wissen müssten, wie hart der Markt in der Computerbranche umkämpft ist. Neben einer hohen Hardware-Qualität ist Kundenvertrauen immens wichtig. Zwei der weltweit bekanntesten und erfolgreichsten Computerhersteller \cite[vgl.]{pc_hersteller} haben den Faktor "Vertrauen" nicht hinreichend erfüllt. Denn sowohl Lenovo als auch DELL haben das Vertrauen ihrer Kunden stark missbraucht. 

\subsection{Lenovo}
\subsubsection{Allgemeiner Ablauf}
Durch eine bereits vorinstallierte Software der Firma Superfish hat Lenovo versehentlich eine gravierende Sicherheitslücke auf einigen ihrer vertriebenen Notebookmodelle eingebaut. Dadurch wurde der Kunde einer zusätzlichen Gefahr eines erleichterten Hackerangriffes ausgesetzt.
Mit der Superfish-Software beabsichtigte Lenovo, dem Nutzer gezielt personalisierte Werbung während des Surfens im Internet anzuzeigen. Um dies auch bei verschlüsselten Internetverbindungen (HTTPs) zu ermöglichen, wurde bei der Softwareinstallation auf den neuen Notebooks ein von Superfish selbst erstelltes Root-Zertifikat mit installiert. Der Kunde kaufte somit unwissentlich ein neues Notebook, auf dem ein bereits vor der Auslieferung manipuliertes Windows-Betriebssystem läuft. 
Mit dem selbst signierten Root-Zertifikat wurden sichere (verschlüsselte) HTTPs-Verbindungen in Form einer Man-in-the-Middle Attacke aufgebrochen. Dadurch konnte sämtlicher Datenverkehr mitgelesen und  manipuliert werden. Im Fall von Lenovo geschah dies durch Einblendung von Werbung. 

\subsubsection{Der Angriff und die benötigten Mittel}
Die Internetseite golem.de beschreibt die Idee und den Ablauf von den benutzerspezifischen Werbeeinblendungungen auf den Lenovo-Rechnern wie folgt: „Grundsätzlich ist die Idee von Superfish, dass das Programm Bilder auf Webseiten durchsucht und anhand von Algorithmen versucht zu erkennen, was sich darauf befindet. Auf Basis dessen werden dem Nutzer passende Shopping-Angebote als Werbebanner angezeigt. Geradezu zynisch wirkt die Beschreibung des Lenovo-Angestellten im Forum: Die Funktion diene dazu, Nutzern zu helfen, visuell Angebote für Produkte zu finden, bei denen sie Schwierigkeiten haben, sie mittels einer textbasierten Suchmaschine zu finden.“\cite{superfish}
Für die Umsetzung der Idee reichte es nicht, dass nur ein Programm von Superfish auf den Lenovo-Rechnern installiert wird. Zusätzlich benötigte Superfish unbedingt ein eigenes signiertes Root-Zertifikat auf dem System. Ohne dieses Zertifikat wäre es nicht möglich gewesen, auch in verschlüsselten Internetverbindungen den Suchalgorithmus anzuwenden, um anschließend personalisierte Werbung einzublenden. Mit dem eigenem Root-Zertifikat war Superfish in der Lage, für jede Verbindung, die zu einer HTTPs-Webseite aufgebaut wurde, ein eigenes, gefälschtes Zertifikat dynamisch zur Laufzeit zu erzeugen. Diesem Zertifikat wurde automatisch vertraut, da bereits dem zugehörigen Wurzelzertifikat vertraut wurde. Dem Root-Zertifikat von Superfish wurde wiederum vertraut, da dieses von Superfish direkt im Windows-Zertifikatsspeicher abgelegt wurde. Der Anwender bekam dadurch nicht mit, dass keine direkte verschlüsselte Kommunikation zu dem Server, der die Webseite hostet, aufgebaut wurde. %TODO BILD einfügen
Weiterhin kam erschwerend hinzu, dass das Programm von Superfish für die Man-in-the-Middle Attacken ein schwaches Zertifikat nutzte. Das Root-Zertifikat verwendet für die digitale Signatur einen SHA-1 Hash-Alsgorithmus und für die asymmetrische Verschlüsselung ein 1024 Bit RSA-Verschlüsselungsverfahren. Sowohl der SHA-1 Hash-Algorithmus als auch das RSA-Verschlüsselungsverfahren mit einer Schlüssellänge von 1024 Bit sind bereits erfolgreich geknackt worden und daher als unsicher einzustufen. 
Doch noch fataler als der Einsatz des unsicheren Hash-Algorithmus und des zu schwachen Verschlüsselungsfahrens, ist die Leichtigkeit der Entschlüsselung des privaten, geheimen Schlüssels. Robert Graham beschreibt in seinem Blog auf Errata Security eindrucksvoll, wie mit simplen Mitteln der Private Key exportiert und anschließend mit einer einfachen Wörterbuch-Attacke entschlüsselt werden konnte. \cite[vgl.]{certificate_ex}
Durch den privaten Schlüssel, welcher für alle betroffenen Geräte identisch ist, ist jeder Angreifer in der Lage, genau wie das Superfish-Programm, eigene „vertrauenswürdige“ Zertifikate für bösartige Verwendungen zu erstellen. Dadurch, dass den Angreifer-Zertifikaten ebenfalls automatisch vertraut wird, bekommen Anwender nicht mit, dass sie z. B. auf einer gefälschten Webseite surfen und ihre Daten ausspioniert werden. Da das Superfish Root-Zeritifikat im Windows-Zertifikatsspeicher abgelegt wurde, können Angreifer mit Zertifikaten, die durch dieses Root-Zertifikat signiert wurden, nicht nur bösartige Webseiten, sondern auch kriminelle Software (z. B. Malware) dem Nutzer problemlos als gutartig erscheinen lassen.

\subsection{DELL}
Der Computerhersteller DELL leistete sich einen ähnlich gravierenden Sicherheitsfehler wie sein Konkurrent Lenovo zuvor. Genau wie Lenovo hat DELL auch selbst signierte Root-Zertifikate auf einigen seiner Laptops installiert. Bei dem US-amerikanischen Hersteller sind es sogar zwei Root-Zertifikate. Sowohl das eDellRoot, als auch das DSDTestProvider Zertifikat wurden, genau wie das Superfish-Zertifikat, im Windows-Zertifikatsspeicher abgelegt. Beim Aufruf der allgemeinen Eigenschaften des eDellRoot-Zertifikats wird sogar ein Hinweis angezeigt, dass ein passender Private Key vorhanden ist. Joe Nord wendet in seinem Online-Blog die gleichen Vorgehensweisen zum Export und zur Entschlüsselung des DELL Private Keys an, wie Robert Graham beim Superfish Private Key. %TODO BILD einfügen
Angreifer mit dem privaten DELL Schlüssel haben die gleichen Angriffsmöglichkeiten auf infizierte DELL Geräte, wie Angreifer mit dem Superfish Private Key auf infizierte Lenovo Geräte.
Beide DELL Zertifikate wurden durch Software installiert, das eDellRoot Zertifikat mit dem Dell Foundation Services Programm und das DSDTestProvider Zertifikat mit der Dell System Detect Software. 
Sie sollen, laut DELL, für einfacheren Support dienen. 
DELL hat auf seiner Webseite folgende Stellungnahme publiziert: „Das eDellRoot Zertifikat wurde durch eine Dell Foundation Services Anwendung auf Ihrem Dell PC installiert und wird vom Support verwendet, um einen besseren, schnelleren und einfacheren Support für unsere Kunden bereitzustellen. Das Zertifikat ist keine Malware oder Adware. Das Auslesen der Service-Tag-Nummer des Systems im online Support von Dell, ermöglicht uns, schnell das Computermodell zu erfassen. Dies macht es leichter und schneller den Service unserer Kunden zu identifizieren. Dieses Zertifikat wird nicht zum Sammeln von persönlichen Kundendaten verwendet. Es ist auch wichtig zu beachten, dass das Zertifikat sich nicht selbst wieder installiert, sobald es mit dem, von Dell empfohlenen, Prozess ordnungsgemäß entfernt wurde. %TODO dell Zertifikat einfügen (http://www.dell.com/support/article/us/en/19/SLN300321/DE)
Wir haben alle unsere Anwendungen geprüft und können bestätigen, dass keine anderen Stammzertifikate werksseitig installiert sind. Wir haben herausgefunden, dass das Dell System Detect und sein DSDTestProvider Root-Zertifikat ähnliche Eigenschaften wie das eDellRoot haben. Im Falle vom Dell System Detect wird die Software vom Kunden heruntergeladen, um proaktiv mit der Dell Support-Website zu arbeiten. Wie schon eDellRoot, wurde dieses Zertifikat entwickelt, um einen schnelleren, personalisierten und einfacheren Support für unsere Kunden zu liefern. Das Problem der Zertifikate betrifft nur Kunden, welche die Software Funktionalität auf unserer Support-Website zwischen dem 20. Oktober und dem 24. November genutzt haben. Die Anwendung wurde umgehend von der Website entfernt und eine neue Version, ohne Zertifikat, ist nun verfügbar. Wir arbeiten an einem Software Update, um das Problem zu beheben und zeigen Ihnen nachfolgend, wie Sie das Zertifikat entfernen können.“\cite{dell}
DELL hat mit seiner Aussage offiziell bestätigt, eigene Zertifikate auf Nutzer-Systemen installiert zu haben. Das eDellRoot wurde augenscheinlich bereits vor Auslieferung auf den Geräten installiert und das DSDTestProvider erst nachträglich bei dem Besuch der DELL Support-Webseite. Es fällt auf, dass das eDellRoot Zertifikat sich anscheinend erneut installiert, wenn es nicht mit spezieller DELL Software ordnungsgemäß entfernt wurde. 